% TCM: TOTEM Communication Middleware
% Copyright: Copyright (C) 2009-2012
% Contact: denis.conan@telecom-sudparis.eu, michel.simatic@telecom-sudparis.eu
% Permission is granted to copy, distribute and/or modify this document
% under the terms of the GNU Free Documentation License, Version 1.3
% or any later version published by the Free Software Foundation;
% with no Invariant Sections, no Front-Cover Texts, and no Back-Cover Texts.
% A copy of the license is included in the section entitled "GNU
% Free Documentation License".

\section{Installation of RabbitMQ}
\label{S_installation_rabbitmq_local}

In this section, we provide the procedure for installing
\textsf{RabbitMQ}: the broker, the Java client for Android phones, the
Python client for the web framework, and the JavaScript client for the
Browser. Similar instructions are included in a separate section,
namely Section~\ref{SS_ec2_configure_instance}, for installing
\textsf{RabbitMQ} on Amazon EC2 cloud.

\subsection{RabbitMQ broker on desktop computers}

In this section, we install the software from a regular archive so
that we will execute \textsf{RabbitMQ} broker as a non-\texttt{root}
user. The configuration of the environment
variables \begin{small}\texttt{RABBITMQ\_MNESIA\_BASE}\end{small}
and \begin{small}\texttt{RABBITMQ\_LOG\_BASE}\end{small} is important
to allow executing the broker as a non-\texttt{root} user. This is
done as follows.
\begin{enumerate}
\item The instructions we provide are extracted from the ``install''
  Web page at this
  URL: \begin{small}\texttt{http://www.rabbitmq.com/\-install.html}\end{small}. Refer to this Web page when necessary.
\begin{itemize}
\item At first, install Erlang and Python 2.6:
\begin{itemize}
\item On GNU/Linux, Debian distribution, check the versions that are
  installed, executing the following commands:
\begin{shellcmd}
\$ erl
Erlang R13B03 (erts-5.7.4) [source] [smp:2:2] [rq:2] [async-threads:0] [hipe] [kernel-poll:false]

Eshell V5.7.4  (abort with ^G)
...
\$ python --version
Python 2.6.6
\end{shellcmd}
\item Or execute the following commands as the user \textsf{root} or
  as a \textsf{sudouser}:
\begin{shellcmd}
apt-get install erlang
apt-get install python
\end{shellcmd}
\end{itemize}
\item Next, install \textsf{RabbitMQ} version~2.7.1 from the ``Package
  for generic Unix systems'' (file
  \textsf{rabbitmq-server-generic-unix-2.7.1.tar.gz}) or from the
  ``Windows Bundle'' (file
  \textsf{rabbitmq-server-windows-2.7.1.zip}). In the following, adapt
  the commands if you are running a Windows operationg system (cf.~the
  Web
  page \begin{small}\texttt{http://\-www.\-rabbitmq.\-com/\-install.\-html\#\-windows}\end{small}).
\begin{itemize}
\item On GNU/Linux, Debian distribution, execute the following commands:
\begin{shellcmd}
wget http://www.rabbitmq.com/releases/rabbitmq-server/ \textbackslash
     v2.7.1/rabbitmq-server-generic-unix-2.7.1.tar.gz
tar xfz rabbitmq-server-generic-unix-2.7.1.tar.gz
\end{shellcmd}
\end{itemize}
\end{itemize}
On Unix-like operating systems, the broker of \textsf{RabbitMQ},
called the server in \textsf{RabbitMQ} terminology, is now installed
in the directory
\textsf{\textsc{\$\{yourdirectory\}}/rabbitmq\_server-2.7.1}.  The
commands of \textsf{RabbitMQ} (\textsf{rabbitmq-server} to launch the
broker and \textsf{rabbitmqctl} to control the broker) are available
in the directory
\textsf{\textsc{\$\{yourdirectory\}}/rabbitmq\_server-2.7.1/sbin}.

\item

\begin{enumerate}
\item On Unix-like operating systems, in order to add the
  \textsf{RabbitMQ} commands to the shell path, and to configure the
  location of the database and the log files, add the following
  commands to the shell script that is executed when opening a shell
  connection, for instance in the file \textsf{\string~/.bashrc}:
\begin{shellcmd}
PATH=\${PATH}:\textsc{\$\{yourdirectory\}}/rabbitmq\_server-2.7.1/sbin
export RABBITMQ\_MNESIA\_BASE=\textsc{\$\{yourdirectory\}}/rabbitmq\_server-2.7.1/mnesia
export RABBITMQ\_LOG\_BASE=\textsc{\$\{yourdirectory\}}/rabbitmq\_server-2.7.1/log
\end{shellcmd}
Note that we assume that you are now going to create the directories
\textsf{\textsc{\$\{yourdirectory\}}/rabbitmq\_server-2.7.1/mnesia}
and \textsf{\textsc{\$\{yourdirectory\}}/rabbitmq\_server-2.7.1/log}.

\item On Windows systems, follow the instructions in \begin{small}\texttt{http://\-www.\-rabbitmq.\-com/\-install.\-html\#\-windows}\end{small}, that is:
\begin{itemize}
\item Set \textsf{\textsc{erlang\_home}} environment variable to where
  you actually put your Erlang installation,
  e.g. \textsf{C:\textbackslash{}Program
    Files\textbackslash{}erl5.7.4} (full path). The \textsf{RabbitMQ}
  batch files expect to execute
  \textsf{\textsc{\%erlang\_home}\%\textbackslash{}bin\textbackslash{}erl.exe}.
\item Create a system environment variable
  (e.g. \textsf{\textsc{rabbitmq\_server}}) for
  \textsf{C:\textbackslash{}Program
    Files\textbackslash{}RabbitMQ\textbackslash{}rabbitmq\_server-2.7.1}.
  Adjust this if you put \textsf{rabbitmq\_server-2.7.1} elsewhere.
\item Append the literal string
  \texttt{;\textsc{\%rabbitmq\_server}\%\textbackslash{}sbin} to your
  system path (as known as \texttt{\%PATH\%}).
\end{itemize}
Note: On Windows systems, it is not necessary to create and set the
variables \texttt{\textsc{rabbitmq\_mnesia\_base}} and
\texttt{\textsc{rabbitmq\_log\_base}}.
\end{enumerate}

\item
On Unix-like and Windows operating systems, your \textsf{RabbitMQ}
installation can be tested by launching the broker as follows:
\begin{shellcmd}
\$ rabbitmq-server -detached \# execute in the background
Activating RabbitMQ plugins ...
0 plugins activated:
\$ rabbitmqctl status
Status of node rabbit@rabbit ...
[{running\_applications,[{rabbit,"RabbitMQ","2.7.1"},
                        {mnesia,"MNESIA  CXC 138 12","4.4.14"},
                        {os\_mon,"CPO  CXC 138 46","2.2.5"},
                        {sasl,"SASL  CXC 138 11","2.1.9.2"},
                        {stdlib,"ERTS  CXC 138 10","1.17"},
                        {kernel,"ERTS  CXC 138 10","2.14"}]},
 {nodes,[{disc,[rabbit@rabbit]}]},
 {running\_nodes,[rabbit@rabbit]}]
...done.
\$ rabbitmqctl stop
Stopping and halting node rabbit@rabbit ...
...done.
\$ rabbitmqctl status
Status of node rabbit@rabbit ...
Error: unable to connect to node rabbit@rabbit: nodedown
diagnostics:
- nodes and their ports on rabbit: [{rabbitmqctl2776,45667}]
- current node: rabbitmqctl2776@rabbit
- current node home dir: /home/bitnami
- current node cookie hash: 58UmUjslvHMdJuJyRDcEag==
\end{shellcmd}
\end{enumerate}

\subsection{RabbitMQ producers and consumers on J2SE applications,
  dedicated RabbitMQ Java client library}
\label{S_rabbitmq_android_phones}

The library for Java producers and consumers of
\textsf{RabbitMQ}, called clients in \textsf{RabbitMQ} terminology, is 
downloaded using Maven. To test the library with your environment, you can do the following:
\begin{itemize}
\item At first, install Sun JDK 1.6 and Maven 2:
\begin{itemize}
\item On GNU/Linux, Debian distribution, check the versions that are
  installed, executing the following commands:
\begin{shellcmd}
\$ java -version
java version "1.6.0_24"
...
\$ javac -version
javac 1.6.0_24
\$ mvn -version
Apache Maven 2.2.1 (rdebian-4)
Java version: 1.6.0_24
Java home: /usr/lib/jvm/java-6-sun-1.6.0.24/jre
\end{shellcmd}
\item Or execute the following commands as the user \textsf{root} or
  as a \textsf{sudoer}:
\begin{shellcmd}
apt-get install sun-java6-jdk
apt-get install maven2
\end{shellcmd}
\end{itemize}
\item Download the directory  \begin{small}\texttt{Totem.CommunicationMiddleware/Sources/TutorialExamples/J2SE/Step1}\end{small}
from the TOTEM Redmine Subversion repository.
\item Launch the RabbitMQ broker, and the J2SE consumers and producers using the script \textsf{run.sh}.
\end{itemize}

\subsection{RabbitMQ producers and consumers on desktop computers,
  Python RabbitMQ client library named Pika}

To limit the number of commands requiring \texttt{root} privileges, we
propose to not install the Python packages using the \texttt{pip} or
\texttt{easy\_install} utility tools. But, of course, you can prefer
performing this installation differently using Python utility
tools. In addition, we only provide the commands for Unix-like
operating systems.

\begin{enumerate}
\item Install the software Pika, version 0.9.5 by executing the following commands
\begin{shellcmd}
\$ cd \textsc{\$\{yourdirectory\}}
\$ wget http://pypi.python.org/packages/source/p/pika/pika-0.9.5.tar.gz
\$ tar xfz pika-0.9.5.tar.gz
\end{shellcmd}
The Python packages Pika is installed in the directory
\textsf{\textsc{\$\{yourdirectory\}}/pika-v0.9.5}.  Insert this
package in the path of Python by adding the directories in the shell
variable \textsf{PYTHONPATH}:
\begin{shellcmd}
export PYTHONPATH=\$PYTHONPATH:\textsc{\$\{yourdirectory\}}/pika-v0.9.5
\end{shellcmd}
\item Check that Pika is correctly installed by trying the following
  Python commands that demonstrate that the Pika package can be
  imported in a Python script:
\begin{shellcmd}
\$ python
Python 2.6.5 (r265:79063, Apr 16 2010, 13:09:56) 
[GCC 4.4.3] on linux2
Type "help", "copyright", "credits" or "license" for more information.
>>> import pika
>>> quit()
\end{shellcmd}
\end{enumerate}

To test the library with your environment, you can do the following:
\begin{itemize}
\item Download the
  directory \begin{small}\texttt{Totem.CommunicationMiddleware/Sources/TutorialExamples/Pika/Step5}\end{small}
  from the TOTEM Redmine Subversion repository.
\item Start the \textsf{RabbitMQ} server with the following command:
\begin{shellcmd}
\$ rabbitmq-server
Activating RabbitMQ plugins ...
0 plugins activated:
...
\end{shellcmd}
\item Check that there is no exhange named \textsf{topic\_logs} by
  executing the following commands:
\begin{shellcmd}
\$ rabbitmqctl list\_exchanges
Listing exchanges ...
amq.direct	direct
amq.topic	topic
amq.rabbitmq.log	topic
amq.fanout	fanout
amq.headers	headers
	direct
amq.match	headers
...done.
\$ rabbitmqctl list\_bindings
Listing bindings ...
...done.
\end{shellcmd}
\item Launch the consumers and the producers using the script
  \texttt{run.sh}.
\end{itemize}


\subsection{RabbitMQ JavaScript producers and consumers using
JavaScript AMQP library}
\label{javascript-node.js}

The library for Javascript AMQP producers and consumers works with
\textsf{Node.js}, which is an event-driven I/O server-side JavaScript
environment. Here follows an installation procedure for Unix-like
operating systems which should be performed on the server side:

\begin{enumerate}
\item Install \textsf{Node.js}:

\begin{itemize}
\item The prerequisites to build \textsf{Node.js} from source are:
\begin{itemize}
\item \textsf{python}: version~2.4 or higher. The build tools distributed with 
\textsf{Node.js} run on python.
\item \textsf{libssl-dev}: If you plan to use SSL/TLS encryption in
  your networking, you need this library. \texttt{libssl} is the
  library used in the \texttt{openssl} tool.
\end{itemize}
\item To download and build \textsf{Node.js}, version~0.4.10 execute the 
following commands:

\begin{shellcmd}
\$ mkdir \textasciitilde/node.js
\$ cd 	\textasciitilde/node.js
\$ wget http://nodejs.org/dist/node-v0.4.10.tar.gz
\$ tar xfz node-v0.4.10.tar.gz
\$ rm node-v0.4.10.tar.gz
\$ cd node-v0.4.10
\$ ./configure --prefix=\$HOME/node.js/node-v0.4.10
\$ make
\$ make install
\$ echo "export PATH=\$HOME/node.js/node-v0.4.10/bin:\$PATH" >> \textasciitilde/.bashrc
\$ echo "export NODE_PATH=\$HOME/node.js/node-v0.4.10:\$HOME/\textbackslash
  node.js/node-v0.4.10/lib/node_modules" >> \textasciitilde/.bashrc
\$ source \textasciitilde/.bashrc
\end{shellcmd}
\item Check that \textsf{node.js} is properly installed by trying the following
command:
\begin{shellcmd}
\$ node -v
v0.4.10
\end{shellcmd}
\end{itemize}

\item Install \textsf{npm}, the package manager for installing additional 
\textsf{Node.js} libraries:
\begin{itemize}
\item If the \textsf{curl} command is not present, you should first install it:
\begin{shellcmd}
\$ sudo apt-get install curl
\end{shellcmd}
\item Then, install \textsf{npm}:
\begin{shellcmd}
\$ curl http://npmjs.org/install.sh | sh
\end{shellcmd}
\end{itemize}

\item Install \textsf{node-amqp} \textit{(version 0.1.0)}, an AMQP client for \textsf{Node.js} by 
executing the following command:
\begin{shellcmd}
\$ npm install -g amqp@0.1.0
\end{shellcmd}

\item Install \textsf{node-mxlrpc}, an XMLRPC client for \textsf{Node.js} by 
executing the following command:
\begin{shellcmd}
\$ npm install -g xmlrpc
\end{shellcmd}
\end{enumerate}

To test the AMQP library with your environment, you can do the following:
\begin{itemize}
\item Download the
  directory \begin{small}\texttt{Totem.CommunicationMiddleware/Sources/TutorialExamples/Javascript}\end{small}
  from the TOTEM Redmine Subversion repository.
\item Enter the \begin{small}\texttt{Step1}\end{small} folder, and launch 
the consumer and the producer using \texttt{run.sh}.\bigskip 
\item At the end of the execution of the script \texttt{run.sh} ,
  check that the output looks like this one:
\begin{shellcmd}
[...]
Starting node rabbit@rabbit ...
...done.
Producer connected to broker.
queue declared.
message published.
Listing queues ...
hello_queue	false	false
...done.
Consumer connected to broker.
Message received: Hello world!
Stopping and halting node rabbit@rabbit ...
...done.
END OF HELLOWORLD TUTORIAL TEST
\end{shellcmd}
\note{Warning:}{ Using the script to launch the consumer and the
  producer will delete every queues and exchanges located on your
  broker. If you don't want to do so, please refer to the
  \texttt{readme.txt} file.}
\end{itemize}

\endinput
