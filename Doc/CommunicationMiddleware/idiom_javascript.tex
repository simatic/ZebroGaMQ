% TCM: TOTEM Communication Middleware
% Copyright: Copyright (C) 2009-2012
% Contact: denis.conan@telecom-sudparis.eu, michel.simatic@telecom-sudparis.eu
% Permission is granted to copy, distribute and/or modify this document
% under the terms of the GNU Free Documentation License, Version 1.3
% or any later version published by the Free Software Foundation;
% with no Invariant Sections, no Front-Cover Texts, and no Back-Cover Texts.
% A copy of the license is included in the section entitled "GNU
% Free Documentation License".

\subsection{JavaScript idioms for game master and spectator applications}
\label{SS_api_JavaScript}

\textit{The idioms are the same for game master and spectator applications. 
Here, we just provide the idioms for the game master application.}

\subsubsection{\texttt{master.js} file}

The master.js file represents the game master application, and uses
the TOTEM API in JavaScript. It consists in the following steps:

\begin{enumerate}

\item Loggin to the game server, and creation of a game instance by an 
XMLRPC to the method \texttt{createGameInstance}.
\item Creation the game logic state managed by the client application:
\begin{small}
\begin{alltt}
gameMasterState = new State(``login'',''password'',''gameName'',''instanceName'');
\end{alltt}
\end{small}
\item Implementation of the \texttt{onExitConnection()} function triggered by 
the API when the connection exits:
\begin{small}
\begin{alltt}
function onExitConnection()\{
  // update the display to show the end screen
  showEnd(); 
\}
\end{alltt}
\end{small}
\item Publish the message for announcing the joigning to
  the game instance broker, the second argument being the content body
  of the message (\emph{e.g.}, \texttt{lisa,mygamename,myinstancename}):
\begin{small}
\begin{alltt}
publishToGameLogicServer(gameMasterState, "join.joinMaster", "lisa,Tidy-City,Instance-1");
\end{alltt}
\end{small}
\end{enumerate}

\subsubsection{State machine actions for the game logic}

The state machine for the game logic is specified into actions kinds and 
actions. Actions kinds can be considered as name spaces to facilitate usage 
of actions. Listing~\ref{L-javascriptactionkinds} gives an example. Please refer to section \ref{SSS_actions} for further details.


\begin{lstlisting}[float=htbp,frame=bt,basicstyle=\scriptsize\sffamily,numbers=left,
   numberstyle=\tiny, stepnumber=1,
    numbersep=5pt,language=python,label=L-javascriptactionkinds,caption=GameMaster's actions of the first action kinds]
function MyFirstActionKind () {}

MyFirstActionKind.prototype.myFirstAction = function (state, publisher, consumer, message){
        println("My first action","Message received from "+publisher+": "+message);
};
\end{lstlisting}

Such an action kind have to be declared to the game logic state:
\begin{small}
\begin{alltt}
gameMasterState.listOfActions.myFirstActionKind = new MyFirstActionKind();
\end{alltt}
\end{small}

\subsubsection{Game logic's State class}

The game logic's State class is defined in the TOTEM library. All the attributes presented in
Listing~\ref{L-gamemasterstatejs} are public.

\begin{lstlisting}[float=htbp,frame=bt,basicstyle=\scriptsize\sffamily,numbers=left,
   numberstyle=\tiny, stepnumber=1,
    numbersep=5pt,language=python,label=L-gamemasterstatejs,caption=GameMaster's state]
function State(login, password, gameName, instanceName, heartbeat, maxRetry){
    this.login = login;
    this.password = password;
    this.gameName = gameName;
    this.instanceName = instanceName;
    if (typeof heartbeat == "number"){
        this.heartbeat = heartbeat;
    }else{
        this.heartbeat = DEFAULT_HEARTBEAT_TASK_PERIOD;
    }
    if (typeof maxRetry == "number"){
        this.maxRetry = maxRetry;
    }else{
        this.maxRetry = DEFAULT_MAX_RETRY;
    }
    this.listOfActions = {  join: new JoinAction(),
                            presence: new PresenceAction(),
                            lifecycle: new LifeCycleAction()};
    [...]
}
\end{lstlisting}

\endinput
